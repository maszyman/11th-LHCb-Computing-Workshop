1

I would like to present recent femtoscopic from the ALICE experiment.

2
Femtoscopy that is to say the method of two-particle correlations at low relative momenta, allows for extracting the space-time characteristics at freeze-out of the source emitting particles, created in collision.
Correlations of identical pions are usually used to perform this study. However, thanks to the significant particle production in Pb-Pb collisions at LHC energy  and excellent capabilities of particle identification in the ALICE experiment, we are able to obtain the full set of femtoscopic results of pions, kaons and protons.
It provides the access to wide range of the transverse mass. Therefore, we have a possibility to test the dynamic evolution of the system created in heavy-ion collision. In particular, we can check whether hydrodynamic collectivity prediction includes mesons as well as baryons.

Furthermore, ALICE collected data from pp and pPb collisions which gives the unique opportunity to compare these three systems at comparable multiplicity of the collision.
The A-A pion femtoscopy results are interpreted within the hydrodynamic framework as a signature of collective
radial flow. Models including this effect are able to reproduce
the ALICE data for central collisions [22,23]. Attempts
to describe the pp data in the same framework have not
been successful so far and it is speculated that additional
effects related to the uncertainty principle may play a role
in such small systems [24]. In p-A collisions, hydrodynamic
models that assume the creation of a hot and dense system
expanding hydrodynamically predict system sizes larger than
those observed in pp, and comparable to those observed in
lower-energy A-A collisions at the same multiplicity [24,25].

Alternatively, a model based on gluon saturation suggests
that the initial system size in p-A collisions should be similar
to that observed in pp collisions, at least in the transverse
direction

the pp, pPb collision system, the analysis is complicated by the presence of sizable background correlation structures in addition to the femtoscopic signal

Typically, two-pion QS correlations are used to extract the characteristic radius of the source [21???27]. However,
higher-order QS correlations can be used as well [28???32]. The novel features of higher-order
QS correlations are extracted using the cumulant for which all lower order correlations are re-
moved [33, 34]. The maximum of the three-pion cumulant QS correlation is a factor of two
larger than for two-pion QS correlations [33???36]. In addition to the increased signal, three-pion
cumulants also remove contributions from two-particle background correlations unrelated to
QS (e.g. from mini-jets [24, 26]). The combined effect of an increased signal and decreased
background is advantageous in low multiplicity systems where a substantial background exists.
Another interesting, although less studied, aspect of QS correlations is the possible suppression due to coherent pion emission
[4???7]. Coherent emission may arise for several reasons such as from the formation of a disori-
ented chiral condensate (DCC) [8???11], gluonic or pionic Bose-Einstein Condensates (BEC)
[12???15], or multiple coherent sources from pulsed radiation [16].

3
The mutual two-particle correlation comes from
the (anti-)symmetrization of the wave function for pairs of identical particles,
or from the Final State Interaction (FSI) for charged or neutral hadrons.
The femtoscopic correlation function is expressed mathematically as

. The aim of the femtoscopic analysis is to use
a measured C(q) and a known form of ?? to deduce information about S

h the functional
form of S, which in heavy ion collision analysis is usually assumed to be a
three-dimensional ellipsoid with a Gaussian density profile
For heavier particles, a simplified analysis is done in only one dimension
in PRF. S is then assumed to be
S(r) ??? exp 
???
r
???2
4R2
inv

, (5)
where Rinv is the single-particle direction-averaged source size

However, the full
three-dimensional analysis requires significant statistics, which is often not
available, particularly for heavier particles. Then, the correlation is represented
in one dimension only, as a function of q = |q|. The decomposition
of the relative momentum into components is most often done in the Longitudinally
Co-Moving System (LCMS), a reference frame in which the total
longitudinal (along the beam axis) momentum of the pair vanishes. The
three directions are then: longitudinal, outwards (along the pair transverse
momentum) and sidewards (perpendicular to the other two).

4

We are doing the measurements with the ALICE detector. 
I just want to mention here that I will show results from pp collisions at 7 TeV as well as Pb-Pb collisions at 2.76 TeV. (For the particle identification we use Time Projection Chamber and Time-of-Flight detector.) 
More details of femtoscopic analyses for kaons and baryons can be found during the poster session.

5

In Fig. 1 the projections
of three example correlation functions along these axes are
shown. A significant, approximately Gaussian enhancement
at low relative momentum is seen in all projections. The width
of the correlation grows with increasing centrality (lowering
multiplicity) as well as with increasing k T .
For like-sign pions the strong-interaction contribution is
small for the source sizes expected here (a few fm) [42],
so it is neglected. The remaining  is a convolution of
the Coulomb interaction and wave-function symmetrization.
As an approximation, the Coulomb part is factored out and
integrated separately in the procedure known as the Bowler???
Sinyukov fitting [48,49]. It is well tested and is applicable for
pions and for the large source sizes expected in this analysis.

6

The radii in all directions
are in the range of 2 to 8.5 fm. The radii universally decrease
with increasing k T , in qualitative agreement with a decreasing
homogeneity length, as predicted by hydrodynamics. Such
behavior is a strong indication of a large degree of collectivity
in the created system. The radii are also universally higher for
more central collisions, which correspond to growing final-
state event multiplicity. For the lowest k T , R long is generally
the largest, whereas at large k T there is no universal ordering
of the radii.
the femtoscopic volume
scales with the final-state event multiplicity, and that each of
the three-dimensional radii separately scales with this value
taken to the power 1/3. In Fig. 4 we present the dependence
of the radii on multiplicity for Pb-Pb collisions. The scaling is
evident for all datasets, for all three directions, and all analyzed
pair momentum ranges.
Similarly, hydrodynamics predicts approximate scaling of
the radii with pair transverse mass m T = (k T 2 + m 2 ?? ) 1/2 [35].
The slope parameters of the lines shown in Fig. 4 are plotted
in Fig. 5 as a function of m T . They are fit with a power-law
function of the form
The dependence of
the values of femtoscopic radii on centrality and k T factorizes
into a linear dependence on dN ch /d?? 1/3 and a power-law
dependence on m T .

7

Correlations of identical pions are usually used to perform this study. However, thanks to the significant particle production in Pb-Pb collisions at LHC energy  and excellent capabilities of particle identification in the ALICE experiment, we are able to obtain the full set of femtoscopic results of pions, kaons and protons.
It provides the access to wide range of the transverse mass. Therefore, we have a possibility to test the dynamic evolution of the system created in heavy-ion collision. In particular, we can check whether hydrodynamic collectivity prediction includes mesons as well as baryons.
What is more, performing analyses for different systems, we are facing different sources of correlations and different combinations of Quantum Statistics and also Coulomb and Strong FSI. (Besides, differences includes the level of systematics, for instance in case of charged and neutral kaons). 

To summarise, the first simultaneous centrality- and mT-differential measurement
of pp and pp femtoscopic correlations in heavy-ion collisions (including LHC experiments)
has been presented. The radii extracted from the one-dimensional pp
and pp correlation functions in combination with the radii obtained from kaon
femtoscopy were found to exhibit the transverse mass scaling within systematic
uncertainties. Such dependence is consistent with the predictions of the hydrodynamic
models with the collective flow. Nonetheless, one can observe that the
pion radii deviate from the exact mT-scaling. This can be understood as a result
of performing the one-dimensional analysis. Indeed, hydrodynamic models
predict the transverse mass scaling of three-dimensional radii calculated in the
LCMS. One-dimensional radii presented in this chapter were obtained in the PRF,
though.

8

In Fig. 6 we show the comparison of the correlation
function at multiplicity 12  Nch  16 in an intermediate
kT range, where the long-range correlations are apparent,
to the Monte-Carlo (MC) calculation. The simulation used
the PYTHIA generator [26], Perugia-0 tune [27] as input. In
this model the enhanced pair production at small relative
angle (which is equivalent to small q in the kT ranges
considered here) is associated with soft parton fragmentation
or mini-jets

9

In Fig. 7 the heavy-ion data from Pb-Pb collisions at the
LHC reported in this work are compared with the previous
measurements, including results obtained at lower collision
energies. It has been argued [20] that three-dimensional
femtoscopic radii scale with the cube root of measured
charged-particle multiplicity not only for a single energy and
collision system, but universally, across all collision energies
and initial system sizes. The dashed lines in the figure are
linear fits to heavy-ion data available before the startup of
the LHC (the dotted lines show one-sigma contours of these
fits). At lower energies the linear scaling was followed well in
long and side directions and only approximately in out, with
some outliers such as the most peripheral collisions reported
by STAR. Our data at higher collision energy show that the
scaling in the long direction is preserved. The data for the side
direction fall below the scaling trend, although still within
the statistical uncertainty. A clear departure from the linear
scaling is seen in the out direction; data from the LHC lie
clearly below the trend from lower energies. Such behavior
was predicted by hydrodynamic calculations [33] and is the
result of the modification of the freeze-out shape

the femtoscopic
radii for pp collisions also exhibit linear scaling in the same
variable, albeit with significantly different parameters. In this
case the scaling between different colliding systems is broken
again in the longitudinal direction

This observation is consistent with an expectation
that the final freeze-out volume, reflected in the femtoscopic
radii, should scale with the final-state observable (such as,
e.g., dN ch /d?? 1/3 ), while the simple ???geometric??? initial-state
variables do not contain enough information

10

So now, the question arises what about pPb.
In p-A collisions, hydrodynamic
models that assume the creation of a hot and dense system
expanding hydrodynamically predict system sizes larger than
those observed in pp, and comparable to those observed in
lower-energy A-A collisions at the same multiplicity [24,25].
However, such models have an inherent uncertainty of the
initial-state shape and size, which can also differ between pp
and peripheral A-A collisions.
Alternatively, a model based on gluon saturation suggests
that the initial system size in p-A collisions should be similar
to that observed in pp collisions, at least in the transverse
direction

11

The radii are in the range of 0.6 to 2.4 fm in all
directions and universally decrease with k T . The magnitude of
this decrease is similar for all multiplicities in the out and long
directions and is visibly increasing with multiplicity in the
side direction. The radii rise with event
??? multiplicity. The plot
also shows data from pp collisions at s = 7 TeV [16] at the
highest multiplicities measured by ALICE, which is slightly
higher than the multiplicity measured for the 20%???40% V0A
signal range in the p-Pb analysis. At small k T , the pp radii
are lower by 10% (for side) to 20% (for out) than the p-Pb
radii at the same dN ch /d?? 1/3 . At high k T the difference in
radius grows for R out , while for R long the radii for both systems
become comparable. The distinct decrease of radii with k T is
observed both in both pp and p-Pb.

Hydrodynamic model calculations for p-Pb collisions
[24,25], shown as lines in Fig. 7, predict the existence of
a collectively expanding system. Both models employ two
initial transverse size assumptions, R init = 1.5 fm and R init =
0.9 fm, which correspond to two different scenarios of the
energy deposition in the wounded nucleon model [25]. The
resulting charged-particle multiplicity densities dN ch /d?? of
45 [25] and 35 [24] are equal to or higher than the one in
the ALICE 0%???20% multiplicity class. The calculations for
R out overestimate the measured radii, while the ones with
large initial size strongly overpredict the radii. The scenarios
with lower initial size are closer to the data. For R side , the
calculations are in good agreement with the data in the highest
multiplicity class, both in magnitude and in the slope of the
k T dependence. Only the Shapoval et al. [24] calculation for
large initial size shows higher values than data. For R long ,
calculations by Bo ??zek and Broniowski [25] overshoot the
measurement by at least 30% for the most central data, while
those by Shapoval et al. are consistent within systematics.
Again, the slope of the k T dependence is comparable. The study
shows that the calculation with large initial size is disfavored
by data. The calculations with lower initial size are closer to
the experimental results, but are still overpredicting the overall
magnitude of the radii by 10%???30%.

The CGC approach has provided a qualitative statement on
the initial size of the system in p-Pb collisions, suggesting
that it is similar to that in pp collisions [27,28]. The measured
radii, at high multiplicities and low k T , are 10%???20% larger
than those observed at similar multiplicities in pp data.
For lower multiplicities the differences are smaller. These
differences could still be accommodated in CGC calculations

is imilar between pp and p-Pb collisions in the side direction.
Another similarity is the distinctly non-Gaussian shape of
the source, which in pp and p-Pb is better described by an
exponential-Gaussian-exponential form. It appears that data
in p-Pb collisions still exhibit strong similarities to results
from pp collisions. However, some deviations, which make
the p-Pb more similar to A-A collisions, are also observed,
especially at high multiplicity. The differences between small
systems such as pp and p-Pb and peripheral A-A data are most
naturally explained by the significantly different initial states
in the two scenarios

12

The novel features of higher-order
QS correlations are extracted using the cumulant for which all lower order correlations are re-
moved [33, 34]. The maximum of the three-pion cumulant QS correlation is a factor of two
larger than for two-pion QS correlations [33???36]. In addition to the increased signal, three-pion
cumulants also remove contributions from two-particle background correlations unrelated to
QS (e.g. from mini-jets [24, 26]). The combined effect of an increased signal and decreased
background is advantageous in low multiplicity systems where a substantial background exists.

The three-particle correlation function
C 3 (p 1 , p 2 , p 3 ) = ?? 3
N 3 (p 1 , p 2 , p 3 )
N 1 (p 1 ) N 1 (p 2 ) N 1 (p 3 )
(6)
is defined as the ratio of the inclusive three-particle spectrum over the product of the inclusive
single-particle
spectra. In analogy to the two-pion case, it is projected onto the Lorentz invariant
q
|~ p
+~ p
+~ p
|
T,3
. The
Q 3 = q 212 + q 231 + q 223 and the average pion transverse momentum K T,3 = T,1 T,2

K 3 (q 12 , q 31 , q 23 ) denotes the three-pion FSI correlation, which in the generalized River-
side (GRS) approach [42, 56, 57] is approximated by K 2 (q 12 ) K 2 (q 31 ) K 2 (q 23 ). It was found to
describe the ?? ?? ?? ?? ?? ??? three-body FSI correlation to the few percent level [42]. From Eq. 7 one
can extract N 3 QS and construct the three-pion QS cumulant correlation

13

the two-particle QS distributions, N 2 QS , and correlations, C 2 QS , are extracted
from the measured distributions in intervals of k T assuming
C 2 (q) = N [(1 ??? f c 2 ) + f c 2 K 2 (q)C 2 QS (q)] B(q) .
(2)
The parameter f c 2 characterizes the combined dilution effect of weak decays and long-lived
resonance decays in the ???core/halo??? picture [50, 51]. In Pb???Pb, it was estimated to be 0.7 ?? 0.05
with mixed-charge two-pion correlations [42]. The same procedure performed in pp and p???
Pb data results in compatible values. The FSI correlation is given by K 2 (q),

The same-charge two-pion QS correlation can be parametrized by an exponential
C 2 QS (q) = 1 + ?? e ???R inv q ,
(3)
as well as by a Gaussian or Edgeworth expansion
2
2
C 2 QS (q) = 1 + ?? E w 2 (R inv q) e ???R inv q
???
?? n
??? H n (R inv q) ,
E w (R inv q) = 1 + ???
n
n=3 n!( 2)
(4)
(5)
where E w (R inv q) characterizes deviations from Gaussian behavior, H n are the Hermite poly-
nomials, and ?? n are the Edgeworth coefficients [54]. The first two relevant Edgeworth coeffi-
cients (?? 3 , ?? 4 ) are found to be sufficient to describe the non-Gaussian features at low relative
momentum.

The three-pion same-charge cumulant correlations are then projected onto 3D pair relative mo-
menta and fit with
c3

14

The absence of two-particle correlations in the three-pion cumulant can be demonstrated via
the removal of known two-body effects such as the decay of K s 0 into a ?? + + ?? ??? pair (Fig. 1).
The mixed-charge three-pion correlation function (C 3 ??????? ) projected onto the invariant relative momentum of one of the mixed-charge pairs in the triplet exhibits the K s 0 peak as expected
around q ????? = 0.4 GeV/c, while it is removed in the cumulant.

15

In Fig. 2 we present three-pion correlation functions for same-charge (top panels) and mixed-
charge (bottom panels) triplets in pp, p???Pb, and Pb???Pb collision systems in three sample mul-
tiplicity intervals. For same-charge triplets, the three-pion cumulant QS correlation (c ??????
) is
3
???????
clearly visible. For mixed-charge triplets the three-pion cumulant correlation function (c 3 ) is
consistent with unity, as expected when FSIs are removed.
Also shown in Fig. 2 are model
calculations of c 3 in PYTHIA (pp), DPMJET (p???Pb) and HIJING (Pb???Pb), which do not con-
tain QS+FSI correlations and demonstrate that three-pion cumulants, in contrast to two-pion
correlations [24, 26], do not contain a significant non-femtoscopic background, even for low
multiplicities.

16

To be independent of the assumed functional form for c 3 , the same-charge three-pion cumulant
correlation functions are directly compared between two collision systems at similar multiplic-
ity. Fig. 6(a) shows that while the three-pion correlation functions in pp and p???Pb collisions
are different, their characteristic widths are similar. It is therefore the ?? e,3 values which differ
the most between the two systems. Fig. 6(b) shows that the correlation functions in p???Pb and
Pb???Pb collisions are generally quite different.

17

The extracted radii in each multiplicity interval and system correspond to different hN ch i values.
To compare the radii in pp and p???Pb at the same hN ch i value, we perform a linear fit to the
pp three-pion Edgeworth radii as a function of hN ch i 1/3 . We then compare the extracted p???Pb
three-pion Edgeworth radii to the value of the pp fit evaluated at the same hN ch i. We find that the
Edgeworth radii in p???Pb are on average 10 ?? 5% larger than for pp in the region of overlapping
multiplicity. The comparison of Pb???Pb to p???Pb radii is done similarly where the fit is performed
to p???Pb data and compared to the two-pion Pb???Pb Edgeworth radii. The Edgeworth radii in
Pb???Pb are found to be on average 45 ?? 10% larger than for p???Pb in the region of overlapping
multiplicity.

18

19

The novel effects measured with three-particle correlations are isolated with the r 3 function [21,22]:
The intercept of r 3 , I , is
expected to be 2 in the case of fully chaotic particle-emitting
sources and less than 2 in the case of partially coherent sources
owing to limited statistical
precision we project r 3 from three-dimensional invariant
relative momenta to one-dimensional Q 3 . A fit quartic and
quadratic in Q 3 is performed,
The coherent
fraction (G) can be extracted from the intercept as

20, 21

We now present r 3 versus Q 3 in Figs. 10 and 11 in six
centrality bins and two K T,3 bins. The data are fit with a
quartic and quadratic fit as shown by Eqs. (11) and (12). The
systematic uncertainties at large Q 3 are typically larger than
50%, while at low Q 3 they are much smaller. At low Q 3 , one
notices that r 3 is further below the chaotic limit (2.0) in Fig. 10
than in Fig. 11.
The largest systematic uncertainty in Figs. 10 and 11 is
attributable to the residual correlation of c ???????

Given the large uncertainties at large Q 3 , r 3 does not change
significantly with centrality and is equally well described by
quartic and quadratic fits

From the intercepts of r 3 at Q 3 = 0 presented in Tables III
and IV, the corresponding coherent fractions (G) may be
extracted using Eq. (13). For low K T,3 , the centrality averaged
intercepts (0%???50%) of r 3 may correspond to coherent frac-
tions of 28% ?? 3% and 24% ?? 9% for quartic and quadratic
intercepts, respectively. For high K T,3 , the corresponding
coherent fractions are consistent with zero for both quartic and
